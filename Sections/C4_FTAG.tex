\chapter{\color{oxfordblue} Flavour Tagging}
\ChapFrame

\textit{
This chapter focuses on an essential task in the ATLAS experiment: identifying particles flying through the detector. This objective of assigning labels is referred to as tagging. An important family of particules to be tagged are quarks, and disentangling which specific flavour of the quarks should be assigned to an observed signal is called flavour tagging. Free quarks and gluons hadronise as per the rules of \gls{qcd}, forming large number of particles that can themselves further decay. Such a dynamic results in a many particles radiating within a clone centred around the inital flavoured particle, a structure referred to as a jet. Consequently, this chapter introduces method to tag jets, to identify the flavour of the initial parton. In particular, the first training of \gls{dl1d} is described in details, as well as the hyperparameter optimisation of \gls{gn2}.
}

\section{Tagging $b$- \& $c$-Jets}
A fundamental ingredient in any ATLAS analysis is the ability to correctly identify particles in the aftermath of a collision, from $\tau$-leptons, to $b$- and $c$-quarks, and gluons $g$. Having well-calibrated and optimally performing b- and c-tagging tools is of primary importance in studies of the Higgs boson couplings to $b$- and $c$-quarks. It is also critical for top $t$-quark measurements and searches for extensions of the \gls{sm}. As described by the theory of \gls{qcd}, colour-charged objects, such as a $b$- or a $c$-quark, undergo hadronisation to form collections of colourless hadrons. These hadrons, mostly $B$ for $b$-quark and $D$ for $c$-quark, are unstable and further decay in the volume of the detector. Such a succession of decays leaves a collection of particles within a cone oriented in the direction of the original parton, an easily recognisable pattern referred to as a \textit{jet}. From an analysis of the complicated structure of the jet, the flavour of the initially decaying particle can be reconstructed. This is the task of \textit{flavour tagging}. In the ATLAS experiment, the tool to achieve this identification, called \textit{tagger}, is developed and maintained centrally by the \gls{ftag}. \\

Tagging $b$-jets benefits from a particularly advantageous configuration:  the $b$ is the lightest element of the third generation and must decay through a flavour-changing process. Because of the \gls{sm} all value of the $V_{bc}$ \gls{ckm} matrix element, this decay process is suppressed, giving $B$-hadrons a characteristically long-lifetime and decay length $(c\tau)_{B} \sim 400$  $\mu$m \cite{Tanabashi:2018oca}. When considering boosted objects with a Lorentz $\gamma$ factor above unity, the location of the $B$-hadron decay, called  \gls{sv}, can be reconstructed by the ATLAS pixel detector \cite{Aad:2019aic}. Other important variables describing the decay of $B$-hadrons are the \glspl{ip} $d_0$ and $z_0$ of charged particles emanating from the \gls{sv}. As shown in Figure \ref{fig:bjet}, $d_0$ and $z_0$ are the transversal and longitudinal distances from the primary vertex to the perigee\footnote{The point of closest approach.} of the track associated with the charged particle. For a $b$-jet, either or both of the \glspl{ip} can be large thanks to the long lifetime of the associated hadron. Other characteristics of $b$-jets are the large number of particles in the final state following their hadronisation, a property described as \textit{high multiplicity} which is due to the large mass of the $b$-quark, and the likely presence of leptons in the jet cone, as 40\% of the decays of $B$-hadrons including either an $e$ or a $\mu$ \cite{Tanabashi:2018oca}. \\

\begin{figure}[h!]
\center
\includegraphics[scale=0.6]{Images/bjet}
\caption{Representation of a $b$-jet.} 
\label{fig:bjet}
\end{figure}

On the contrary, tagging $c$-jets, being at an intermediate scale between light- ($u$, $d$, $s$, and gluons) and $b$-jets, is much more challenging. The decay length for charged (neutral) $D$-hadrons, $(c\tau)_D \sim 300$ $(100)$ $\mu$m \cite{Tanabashi:2018oca}, is smaller than for $B$-hadrons and is difficult to resolve with the currently deployed tracker. The decay chain of $B$-hadrons often includes $D$-hadrons, making a clean separation of $c$-jets from $b$-jets harder. Compared to $b$-jets, $c$-jets have a lower multiplicity which leads to $\tau$-jets being easily mistaken for $c$-jets, as these leptons can hadronically decay into a sufficiently large number of particles to mimic a jet in the detector. For all these reasons, less effort has been historically dedicated to constructing $c$-taggers in ATLAS. The task is however gaining particular attention due to the focus on the challenging $H \rightarrow c\bar{c}$ search \cite{Aaboud:2018fhh}. This chapter presents the development of a novel $b$- and $c$-tagger for the ATLAS experiment.

\section{Flavour Tagging at ATLAS}

In the ATLAS experiment, a choice was made to develop centrally a tagger to be used by the whole collaboration. It relies on a dedicated set of algorithms to perform simultaneously $b$- and $c$-tagging and is continuously improved to meet the requirements of the physics program. Currently, all adopted approaches rely on \gls{dl} methods, given their vastly superior effectiveness. This area of research has been evolving rapidly in recent times due to the community adopting advanced methods from the field of \gls{ai}. As such, various models have been introduced and the last two generations can be split into two categories: 
\begin{enumerate}
  \item The DL1 family are \gls{dl} models built in a hierarchical way. These \gls{dl} methods rely on high-level features reconstructed by sub-algorithms based on physics variables, such as the tracks \glspl{ip}, and the reconstruction of secondary vertices. The most important models in this family are those including a \gls{dl} sub-model to analyse tracks with either a \gls{rnn} approach for \gls{dl1r}, leveraging the \gls{rnnip} sub-tagger, and a DeepSet approaches for \gls{dl1d}, leveraging the \gls{dips} sub-tagger. This last tagger is, at the moment of writing this thesis, the current state-of-the-art deployable tagger for \gls{atlas}.
  \item The GN family of taggers are built on more advanced \gls{dl} method, as they move away from the hierarchical approach of the DL1 family and directly analyse track and jet information in a unique complex architecture. The GN family is based either on a full Graph Attention Network (\gls{gat}) for \gls{gn1}, and a Transformer encoder for \gls{gn2}. This lighter algorithm pipeline greatly simplifies the maintenance and turnaround time for modification, making the process of updating the taggers nimbler and easier to tailor to specific applications. The GN taggers greatly outperform the DL1 family and represent an exciting area of progress for future analysis requiring precise flavour jets tagging. 
\end{enumerate}

\begin{figure}[h!]
  \center
  \includegraphics[scale=0.6]{Images/FTAG/storyFtag.png}
  \caption{Comparison of the performance of \gls{ftag} models introduced through the years. Light and $c$-jet rejection are plotted for different taggers at a fixed $b$-jet tagging efficiency of 70\% on a common $t\bar{t}$ evaluation sets. The multiplicative factors in the bars are with respect to the DL1 model performance.} 
  \label{fig:storyFtag}
\end{figure}

A historical perspective on the evolution of performance for the different taggers mentioned is presented in Figure \ref{fig:storyFtag}, showing a remarkable continuous increase in light- and $c$-jet rejections at a fixed $b$-tagging efficiency of 70\% on a $t\bar{t}$ dataset. The analysis presented in the latter part of this thesis was carried out in the span of 2021-2024, and was therefore restricted to tools and methods available to the experimental team during this period. As such, due to the need to calibrate the GN taggers as explained latter in this chapter, the second family of tagger was not yet ready for deployement. 

\section{DL1 Family of Models: DL1r \& DL1d}
This family of tagger is built with a hierarchical approach, combining low-level algorithms that are independentely optimised into a final \gls{dnn} network of a few layers to output a final prediction. Not all low-level modules are based on \gls{dl}, with some instead directly implementing physics-motivated algorithms. They consist of \cite{Paganini:2289214}:
\begin{itemize}
\item Jet Fitter: a vertexing algorithm based on a Kalman filter to reconstruct the topology and fit the decay chain \gls{pv} $\rightarrow$ $B$ $\rightarrow$ $D$ with the assumption that the vertices of the weakly decaying B/D-hadrons tend to align with the \gls{pv} \cite{ATL-PHYS-PUB-2018-025}. 
\item \gls{sv1}: combining a secondary vertex finder and a tagger to offer flavour discrimination information. The former, based on the VKalVrt vertex reconstruction package \cite{Kostyukhin:685551}, returns a list of candidate secondary vertices with measured quantities assigned to each vertex. The latter derives jet weights based on discriminative variables and computes properties of the \gls{sv}, such as the mass. 
\item \gls{ip} likelihood: IP2D and IP3D are both likelihood-based methods to assign flavour-discriminating weights based, respectively, on the transversal and global impact parameters significance (corresponding to the reweighted \gls{ip} variables by their respective uncertainties) of the tracks \cite{ATLAS:2017bcq}. 
\item Track collection analyser: either with \gls{rnnip} \cite{ATL-PHYS-PUB-2017-003} or \gls{dips} \cite{ATL-PHYS-PUB-2020-014}. These are \gls{dl} approaches to extract discrimination information on the set of tracks associated with a jet. These taggers are further described later in this section.  
\end{itemize}

\begin{figure}[h!]
  \centerline{
  \begin{minipage}[c]{0.4\textwidth}
      \includegraphics[scale=0.5]{Images/Algorithms}
    \end{minipage}
  \begin{minipage}[c]{0.6\textwidth}
      \caption{The algorithms for flavour tagging at ATLAS. High-level taggers are in dark blue, track-based taggers in light blue and vertex-related taggers in white.}
      \label{fig:algo}
    \end{minipage}
  }
\end{figure}

The outputs of these low-level algorithms, as well as certain jet-related variables, such as $p_T$, are then combined as input to a high-level tagger consisting of a fully-connected \gls{nn} called \gls{dl1r} or \gls{dl1d}, respectively if \gls{rnnip} or \gls{dips} is used. The input vector is typically made of 44-45 features. This high-level tagger outputs three probabilities $p_X$ for the analysed jet to correspond to a $b$-, $c$-, or light--flavour (indicated with the letter $u$) such that $p_b + p_c + p_u = 1$. A $b$-tagging score $D_b$ is then derived by computing a scaled log-likelihood ratio: 
\begin{equation}\label{bdisc}
D_b = \log \frac{p_b}{f^b_c \times p_c + (1 - f^b_c) \times p_u},
\end{equation}
where $f^b_c$ is the charm fraction, a parameter that can be modified to tweak the importance of each flavour. The analogous $c$-tagging score $D_c$ is: 
\begin{equation}\label{cdisc}
D_c = \log \frac{p_c}{f^c_b \times p_b + (1 - f^c_b) \times p_u}.
\end{equation}

A jet is $X$-tagged if the $D_X$ discriminant score is above a set threshold constant $c_{wp}$, defining a \textit{Working Point} (WP) with a unique configuration of signal and background (mis-tag) efficiencies. In this context, the efficiency $\epsilon^X_Y$ for $Y$-flavour jets to be $X$-tagged and the corresponding rejection $\mathcal{R}^X_Y$ are respectively defined as:
\begin{equation}
\epsilon^X_Y = \frac{N^{X-tagged}_{Y-jets}}{N_{Y-jets}} \quad \textrm{and} \quad \mathcal{R}^X_Y = \frac{1}{\epsilon^X_Y},
\end{equation}
where $N^{X-tagged}_{Y-jets}$ and $N_{Y-jets}$ are respectively the number of $X$-tagged $Y$-flavoured jets and the total number of $Y$-flavoured jets. \\

These high-level models are trained on \gls{mc} simulated data samples and need to be calibrated on real data to deliver an unbiased estimate, by deriving \glspl{sf} weights correcting the predictions for each jet. Uncertainties are derived on the predicted score and passed along to analyses using the tool. The novel algorithm introduced in this work is the \gls{dl1d} tagger, which relies on the \gls{dips} sub-tagger to extract correlations between the tracks.  

\subsubsection{RNNIP}
The \gls{rnnip} tagger runs on arbitrary-length input sequences made of track features, as ordered by the absolute transverse \gls{ip} significance, to extract tagging information from correlations between tracks \cite{ATL-PHYS-PUB-2017-003}. The vector of track features includes the transverse and longitudinal impact parameter significances, the jet $p_T$ fraction carried by the track, the distance between the track and the jet axis, and a learned 2D embedding of the track quality \cite{Paganini:2289214}. It outputs a probability $p_X$ for the jet to belong to flavour $X$ $\in$ [$b$, $c$, light].

\subsubsection{DIPS}
The \gls{dips} tagger, based on the Deep Sets architecture \cite{NIPS2017_f22e4747}, is a \gls{gnn} approach to model the correlations between an arbitrary number of tracks. Two NNs are trained in this effort. A model $\Phi$ maps each individual track feature vector (similar to \gls{rnnip}) to a latent space (forming the nodes of the graph). The representations of each track in this latent space are then concatenated by a simple summation operation (the edges of the graph) and given as input to a secondary \gls{nn} model $F$ outputting the predicted probability $p_X$ for the jet to belong to flavour $X$ $\in$ [$b$, $c$, light] (the prediction of the graph). This approach has several advantages over \gls{rnnip}, mainly the physically motivated permutation-invariance of the input and the improved training/evaluation time thanks to a more parallelisable architecture. Furthermore, the performance delivered by \gls{dips} is observed to globally outperform \gls{rnnip} \cite{ATL-PHYS-PUB-2020-014}. 

\subsection{Training of DL1d \& DL1r for Run 3}
The ATLAS Collaboration continuously updates its software, updating specific methods to adopt new techniques, maintaining its many tools and adding capabilities. In preparation for the new run of the \gls{lhc} that started in 2021, ATLAS has improved its reconstruction software. As such, important elements used by flavour tagging methods have changed, requiring to retrain all taggers to ensure optimal performance under the new conditions. This work presents the first ATLAS study of the retraining of \gls{dl1r} on the new release (R22) and the first training of \gls{dl1d}, including the \gls{dips} sub-tagger in the complete flavour tagging tools. Other important novelties of this work are the possible inclusion of $\tau$-jets in the model's predictions and a new technique to efficiently process the training data into high statistics dataset using importance sampling. The interest of including $\tau$ stems from their tendency to be miss-classified as $c$-jets when hadronically decaying, as both particles commonly leave three particles in the detector. The resulting taggers are observed to efficiently identify $\tau$-jets thereby providing a new way to perform $\tau$-identification and improving $c$-jet tagging. However, due to the widespread use of the \gls{ftag} algorithms and the difficulties arising in calibrating a tagger with exceptionally good rejection against $\tau$-jets, these are not included in the default version of the tagger nor in the results shown here, but are actively under study for the new generation of tagger in the GN family. \\ 

Two samples from proton-proton collisions at $\sqrt{s} = 13$ are simulated and combine in the datasets:
\begin{itemize}
\item $t\bar{t}$ events, with at least one of the $W$-boson from the top-quark decay further decaying leptonically. This latter constrain simplifies the real data selection when callibrating the tagger.
\item $Z'$ events, where an exotic boson $Z'$ decays as $Z' \rightarrow q\bar{q} \textrm{ or } \tau \bar{\tau}$, with a variable $Z'$ mass to generate a flat $p_T$ spectrum extending the $p_T$-range of the jets studied up to 4 TeV.
\end{itemize}

For both samples, the ATLAS detector is simulated using GEANT4 \cite{Agostinelli:602040} and jets are reconstructed with the anti-kT algorithm with a radius of R = 0.4. For \gls{rnnip}, the tracks considered must pass the following quality selection: $\geq$ 7 hits in the silicon layers, $\leq$ 2 missing hits in the silicon layers, $\geq$ 1 hit in the pixel detector, $\leq$ 1 hit shared by multiple tracks, $p_T$ > 1 GeV, $|d_0|$ < 1 mm, and $|z_0 \textrm{ sin}(\theta)|$ < 1.5 mm. For \gls{dips}, a looser track selection is applied to capture more tracks, modifying the nominal selection in the following way: $p_T$ > 0.5 GeV, $|d_0|$ < 3.5 mm, and $|z_0 \textrm{ sin}(\theta)|$ < 5 mm \cite{ATL-PHYS-PUB-2020-014}. Loosening the selection was the subject of several and studies and was observed to lead to a significant improvement in performance for jets with a $p_T < 250$ GeV for \gls{dips}. From an \gls{ml} viewpoint, a larger set of input information with more noise can still prove benefitial if the underlying model is complex enough to capture useful features in the noisy data, that would otherwise be erased by a more stringent selection. \\

These two samples are combined into a single \textit{hybrid} sample to train the taggers, with 70\% of the total number of jets coming from $t\bar{t}$ and the remaining from the $Z'$. The $t\bar{t}$ and $Z'$ samples cover, respectively, a low- and high-$p_T$ region based on a reconstructed $B$-hadron $p_T$ separation threshold of 250 GeV for $b$-jets and a jet $p_T$ of 250 GeV for non-$b$-jets. The relative weights of the two samples was chosen to avoid having a discontinuity in the $p_T$ spectrum at the junction, as evidenced in Figure \ref{fig:distTraining}. The evaluation of the performance of a trained tagger is performed on the separated sets and unfolded over the flavours.\\

\begin{figure}[h!]
  \center
  \includegraphics[scale=0.29]{Images/FTAG/added/distpt.png}
  \includegraphics[scale=0.29]{Images/FTAG/added/disteta.png}
  \caption{The $p_T$ (left - in MeV) and $|\eta|$ distributions of the resampled $b$-, $c$-, and light-jets in, respectively, blue, orange, and green. The three ses are resampled to have the same $p_T-|\eta|$ 2D distributions, explaining the perfect agreement found in bins of reasonable statistics. The flat $p_T$ spectrum extrending up to several TeV is due to the exotic $Z'$ process generated with varying mass. The large peak at lower $p_T$ is the $t\bar{t}$-process. These sets have 8.3 million jets per flavour.} 
  \label{fig:distTraining}
\end{figure}

ATLAS flavour tagging tools are widely used across the Collaboration. It is therefore essential for the taggers not to learn specific features of the processes simulated  for the training samples but to focus on the inherent differences between the studied flavours and therefore to generalise to other processes. An effective way to limit the importance of the simulated process is to downsample the hybrid sample in [$p_T - \eta$] bins to have the same number of $b$-, $c$-, and light-jets in each 2D bin. This removes the distinction of kinematic phase space between each flavour due to the process specific physics. To avoid biasing the output of the tagger towards the most likely flavours in the process, each jet-flavour is also required to be equally likely in the training set, a requirement satisfied by having the same yield of $b$-, $c$-, and light-jets. Applying this technique, the total statistics available for the R22 training is of $25 \times 10^{6}$ jets per flavour for training. The $t\bar{t}$ and $Z'$ samples for validation and testing are each made of 1 million jets and are not downsampled to have the same [$p_T - \eta$] distribution nor the same yield of different flavours: they represent a realistic distribution of the underlying processes. The main limitator when downsampling are $c$-jets, as all $c$-jets from the $t\bar{t}$ process are selected which limits the amount of $b$- and light-jet that can be taken. This process is extremely wasteful, using only 17\% (11\%) of all available $b$-jets (light-jets) in the $t\bar{t}$ sample.\\

Training is done with the Umami framework \cite{Umami} based on TensorFlow \cite{tensorflow2015-whitepaper} for 300 epochs with a variable learning rate schedule and the default network structure adopted in the previously released \gls{dl1r} (R21): 8 fully connected \gls{nn} of smoothly-decreasing sizes in [256, 128, 60, 48, 36, 24, 12, 6] with \gls{relu} activation leading to a final softmax layer producing the predicted probabilities for each flavour. While the \gls{dips} probabilities used as inputs to \gls{dl1d} come from a model trained on the new release, the \gls{rnnip} probabilities are still from a model trained on the previous one (R21) \cite{ATL-PHYS-PUB-2017-003, ATL-PHYS-PUB-2020-014}. Indeed, due to its significantly lower performance, \gls{rnnip} is no longer supported in the new release and is included for sake of comparability to the previous techniques. The models at an epoch offering the best combined results in terms of $b$-tagging efficiency and rejection from $b$-jets on the validation set are selected for further analysis. Importantly, every training converged to a fixed set of performance values, with no overtraining occurring.\\

Several modifications to the model architecture, list of input variables, and preprocessing and training procedures have been explored, with no significant gain observed:
\begin{itemize}
\item The preprocessing steps were revised to reduce the size of the evaluation sets for the benefit of the training one. A dual approach, downsampling light-jets and upsampling $c$-jets to the $b$-jets [$p_T - \eta$], has also been implemented. This approach uses importance sampling with replacement to obtain the same fraction of the different flavours and the same $p_T$ and $|\eta|$ distribution. While the performance of the majority classes was observed to improve, the efficiency at tagging the upsampled minority class ($c$-jets) was slightly lower. This trade-off can be controlled by modifying the flavour fractions and thus not result in any significant performance changed. This is likely due to the model saturating its performance given the large dataset available.  
\item Several modifications to the list of input features have been attempted, with no clear advantage uncovered. Adding pile-up information (the actual number of interactions per crossing and the number of primary vertices were tested) was not observed to have an impact on the tagging efficiency. Adding other variables from \gls{sv1} or JetFitter was also not observed to improve performance. However, a positive observation is that the IP2D and IP3D taggers can both be safely removed without change to performance, as the information they add is in all likelihood now covered by the \gls{dips} sub-tagger, thereby reducing the list of sub-taggers to maintain.
\item The structure of the network and its training procedure, leveraging transfer learning. Using samples produced with an older release of the ATLAS software to pre-train the model was not observed to deliver a boost in performance when later training on the new release. Changing the size of the network and the batch size was also not observed to have a positive effect.
\end{itemize}

The conclusion driven by the lack of improvements from these three attempts is that models built on this simple \gls{dnn} structure with large dataset are likely saturating their performance from the set of input. The performance of the retrained \gls{dl1r} tagger on the new release was found to be in good agreement with the currently recommended \gls{dl1r}, despite using the same training of \gls{rnnip} on the previous release. In order to establish a meaningful benchmark for the newly trained taggers, the performance of the recommended \gls{dl1r} tagger, trained and evaluated on an analogous set of samples from the previous release (R21), is included in the following results as benchmark under the label \textit{Recom. \gls{dl1r}}. \\

Figure \ref{fig:DL1dtt} presents the \gls{roc} curves on the $t\bar{t}$ (left) and $Z'$ (right) samples for  $b$-tagging. These \gls{roc} plots show, on the $x$-axis, the $b$-tagging efficiency ($\epsilon^b_b$ ) versus, on the $y$-axis, the rejection $\mathcal{R}^b_Y$ for $Y \in$ [$c$, light]. The two bottom sub-plots present the ratio of the c-jet rejection and light-jet rejection curves to the blue ones. This blue curve is the recommended \gls{dl1r} performance and serves as the baseline of the comparison, while the new tagger \gls{dl1d} is plotted in orange. Figure \ref{fig:DL1dz} shows the same plots for $c$-tagging, with respect to $b$- and light-jet rejections. The important observation is the clear gain obtained when replacing \gls{rnnip} with \gls{dips}. Both the $b$- and $c$-tagging performance of \gls{dl1d} clearly dominate the \gls{dl1r} versions, with a significant improvement in background flavour rejection for all tagging efficiency considered, as summarised in Table \ref{tab:max-perf}. The largest improvement in performance is obtained for $b$-tagging on the $t\bar{t}$ process, corresponding to a lower jet momentum. This latter points to a dynamical behaviour of the \gls{dips} subtagger that can be traced back to the looser jet selection. Higher momentum jets are more likely to have a larger set of tracks and these tracks tend to be closer to each other due to relativistic boosting. The looser selection forces the \gls{dips} model enduce to sift through a larger set of noisy tracks which brings lower performance at higher momentum, while a gain is obtained at lower momentum from the nicer geometrical separation and smaller initial set.  \\

In the light-rejection from $b$-jets \gls{roc} curves in Figure \ref{fig:DL1dtt}, there is an elbow in the curve at high $b$-jet efficiency. This effect is also present in the $b$-rejection from $c$-tagging, in Figure \ref{fig:DL1dz}. Both correspond to a set of, respectively, light-jets and $b$-jets that do not overlap with the $b$-jets $b$-tagging and $c$-jets $c$-tagging discriminants distributions, as shown in Figures \ref{fig:scoreDL1dtt} and \ref{fig:scoreDL1dz}. These ``background`` jets are easily removed from the core set of ``signal'' jets due to internal differences between the flavours and the discrete nature of some sub-taggers used. \\

The background rejections of the various taggers for $b$-tagging ($c$-tagging) as a function of the jet transverse momentum $p_T$ at a fixed $b$-efficiency of 70\% ($c$-efficiency of 30\%) per region displayed are shown in Figure \ref{fig:ptDL1dtt} (Figure \ref{fig:ptDL1dz}). Throughout the $p_T$ range considered, \gls{dl1d} outperforms the \gls{dl1r} tagger. The low $p_T$ $b$-rejection from $c$-jets is noticeably better for the retrained tagger compared to \gls{dl1r}. The discontinuity of the rejections between the two processes arises from the inclusive $b$-tagging efficiency being computed inclusively per-region and not exclusively for the whole range. 

%
\begin{center}
\vspace{-1.cm}
\begin{figure}[h!]
\centerline{
\includegraphics[scale=0.45]{Images/FTAG/Reprocessed/plotting_alone/ttbar_comparisons_300.pdf}
\includegraphics[scale=0.45]{Images/FTAG/Reprocessed/plotting_alone/zp_comparisons_300.pdf}
}
\caption{Performance for $b$-tagging with a flavour fraction of $f^b_c = 0.018$. Left: $t\bar{t}$; right: $Z'$. Top: \gls{roc} curves; centre: ratio of $c$-jets rejection from $b$-jets relative to the R22-retrained \gls{dl1r}; bottom: same ratio for light-jets rejection. List of taggers: {\color{blue} recommended \gls{dl1r} from the previous release}; {\color{orange} \gls{dl1d} trained on the new release}; {\color{greenforest} \gls{gn1} test-model trained on the new release}.}
\label{fig:DL1dtt}
\bigskip
\centerline{
\includegraphics[scale=0.45]{Images/FTAG/Reprocessed/plotting_alone_c/ttbar_comparisons_299.pdf}
\includegraphics[scale=0.45]{Images/FTAG/Reprocessed/plotting_alone_c/zp_comparisons_299.pdf}
}
\caption{Performance for $c$-tagging with a flavour fraction of $f^c_b = 0.2$. Left: $t\bar{t}$; right: $Z'$. Top: \gls{roc} curves; centre: ratio of $b$-jets rejection from $c$-jets relative to the R22-retrained \gls{dl1r}; bottom: same ratio for light-jets rejection. List of taggers: {\color{blue} recommended \gls{dl1r} from the previous release}; {\color{orange} \gls{dl1d} trained on the new release}; {\color{greenforest} \gls{gn1} test-model trained on the new release}.}
\label{fig:DL1dz}
\end{figure}
\end{center}
%
\clearpage

\begin{table}[h]
  \begin{center}
      \begin{tabular}{C{2cm}|cc} 
      	 \hline \hline
          \multicolumn{3}{c}{$b$-tagging on $t\bar{t}$} \\ \hline
          WP & $c$-rejection  & light-rejection  \\ \hline
          60\%   & +26\% & +73\% \\ 
          70\%   & +19\% & +56\% \\ 
          77\%   & +12\% & +41\% \\ 
          85\%   & +7\%   & +32\% \\ \hline
          \multicolumn{3}{c}{} \\
           \hline  \hline
           \multicolumn{3}{c}{$c$-tagging on $t\bar{t}$} \\ \hline
          WP & $b$-rejection  & light-rejection  \\ \hline
          25\%   & +26\% & +5\% \\
          30\%   & +25\% & +9\% \\
          40\%   & +22\% & +12\% \\
          50\%   & +18\% & +15\% \\ \hline
      \end{tabular}
      \quad
       \begin{tabular}{C{2cm}|cc} 
       	 \hline  \hline
          \multicolumn{3}{c}{$b$-tagging on $Z'$} \\ \hline
          WP & $c$-rejection  & light-rejection  \\ \hline
          60\%   & +19\% & +43\% \\
          70\%   & +10\% & +32\% \\
          77\%   & +9\%  & +26\% \\
          85\%   & +6\%  & +19\% \\ \hline
          \multicolumn{3}{c}{} \\
           \hline  \hline
           \multicolumn{3}{c}{$c$-tagging on $Z'$} \\ \hline
          WP & $b$-rejection  & light-rejection  \\ \hline
          25\%   & +12\% & +22\% \\
          30\%   & +11\% & +19\% \\
          40\%   & +8\%   & +14\% \\
          50\%   & +7\%   & +10\% \\ \hline  \hline
      \end{tabular}
    \caption{The ratio of background flavour rejection of the \gls{dl1d} to \gls{dl1r} at various tagging efficiencies, both trained on the new release. Top: $b$-tagging ($f^b_c = 0.018$); bottom: $c$-tagging  ($f^c_b = 0.2$); left: $t\bar{t}$; right: $Z'$.}
    \label{tab:max-perf}
  \end{center}
\end{table}

%
\begin{center}
\begin{figure}[h!]
%\vspace{-0.2cm}
\centerline{
\includegraphics[scale=0.5]{Images//FTAG/Reprocessed/plotting_variables/scores_DL1_ttbar_300.pdf}
\includegraphics[scale=0.5]{Images//FTAG/Reprocessed/plotting_variables/scores_DL1_zprime_300.pdf}
}
\caption{Distribution of \gls{dl1d} $b$-tagging discriminant for the different jet flavours, evaluated on $t\bar{t}$ (left) and $Z'$ (right).}
\label{fig:scoreDL1dtt}
\centerline{
\includegraphics[scale=0.5]{Images/FTAG/Reprocessed/plotting_variables_c/scores_DL1_ttbar_c_299.pdf}
\includegraphics[scale=0.5]{Images/FTAG/Reprocessed/plotting_variables_c/scores_DL1_zprime_c_299.pdf}
}
%\vspace{-0.3cm}
\caption{Distribution of \gls{dl1d} $c$-tagging discriminant for the different jet flavours, evaluated on $t\bar{t}$ (left) and $Z'$ (right).}
\label{fig:scoreDL1dz}
\end{figure}
\end{center}
%

\clearpage
%
\begin{center}
\begin{figure}[h!]
\vspace{-0.55cm}
\centerline{
\includegraphics[scale=0.425]{Images//FTAG/Reprocessed/plotting_eff_vs_pt/pT_vs_beff_c_tt_300.pdf}
\includegraphics[scale=0.425]{Images//FTAG/Reprocessed/plotting_eff_vs_pt/pT_vs_beff_u_tt_300.pdf}}
\centerline{
\includegraphics[scale=0.425]{Images//FTAG/Reprocessed/plotting_eff_vs_pt/pT_vs_beff_c_zp_300.pdf}
\includegraphics[scale=0.425]{Images//FTAG/Reprocessed/plotting_eff_vs_pt/pT_vs_beff_u_zp_300.pdf}
}
\caption{Background flavour rejections at a fixed $b$-tagging efficiency of 70\% (per region shown) for the various taggers. Top: $t\bar{t}$; bottom: $Z'$; left: $c$-rejection; right: light-rejection. For each plot, the bottom panel presents the ratio to the recommended \gls{dl1r}.}
\label{fig:ptDL1dtt}
\bigskip
\centerline{
\includegraphics[scale=0.425]{Images//FTAG/Reprocessed/plotting_eff_vs_pt_c/pT_vs_beff_c_tt_299.pdf}
\includegraphics[scale=0.425]{Images//FTAG/Reprocessed/plotting_eff_vs_pt_c/pT_vs_beff_u_tt_299.pdf}}
\centerline{
\includegraphics[scale=0.425]{Images//FTAG/Reprocessed/plotting_eff_vs_pt_c/pT_vs_beff_c_zp_299.pdf}
\includegraphics[scale=0.425]{Images//FTAG/Reprocessed/plotting_eff_vs_pt_c/pT_vs_beff_u_zp_299.pdf}
}
\caption{Background flavour rejections at a fixed $c$-tagging efficiency of 30\% (per region shown) for the various taggers. Top: $t\bar{t}$; bottom: $Z'$; left: $b$-rejection; right: light-rejection. For each plot, the bottom panel presents the ratio to the recommended \gls{dl1r}.}
\label{fig:ptDL1dz}
\end{figure}
\end{center}

In Figures \ref{fig:DL1dtt} and \ref{fig:DL1dz}, a GN-like tagger from the new family base on \gls{gnn} that was in development is introduced: \gls{gn1} \cite{ATL-PHYS-PUB-2022-027}. This model is based on a graph attention network (\gls{gat}) directly processing low-level inputs, thereby diverging from the traditional ATLAS flavour tagging philosophy of combining several low-level sub-taggers into a high-level one, such as in \gls{dl1d}. As exemplified in this plot, the method offer 



\section{The Graph Neural Network family of Tagger}
\gls{gn1} uses the information associated with charged tracks in a jet to directly output the flavour-tag probabilities, which are then combined into analogous discriminants to Equations \ref{bdisc} and \ref{cdisc}. Alongside predicting the flavour of the jet flavour, two auxiliary objectives are also optimised for the training: 
\begin{itemize}
\item the physics process that produced the different tracks and
\item grouping the tracks into vertices.
\end{itemize}
These complementary objectives improve performance and provide useful information on the content of the jets. Thanks to this guided training procedure, \gls{gn1} is able to compete favourably with the higher-level taggers presented so far, with on average a remarkable increase in the background flavour rejection for all tagging efficiencies. Another non-negligible advantage of the \gls{gn1} approach is the simplification of the structure, removing the need to maintain and retune several low-level algorithms. At the time of compiling the results, this tagger is still in development and is included solely to offer an exciting suggestion of the future performance of the tools deployed by the ATLAS flavour tagging group. 

\section{Conclusion}
This chapter introduces the novel tagger \gls{dl1d} to the set of flavour tagging tools available in ATLAS. Based on the \gls{dips} sub-tagger, \gls{dl1d} is found to have improved background rejections at a fixed working point for both $b$- and $c$-tagging. In parallel to the development of this new tagger, the preprocessing pipeline has been revamped and several changes to the input features list and model architecture were explored in order to deliver the best-performing tagger to the collaboration. 
\clearpage


\begin{tcolorbox}[colback=oxfordblue!5,colframe=blue!40!black,title=Summary of the Chapter]
In this chapter, the \gls{ftag} world is introduced. 
\end{tcolorbox}